%-------------------------
% Subject : Resume
% Author : Ganesh Belgur
% Last Updated : 4-January-2022
% Latex Template : Sourabh Bajaj (github.com/sb2nov)
%------------------------

% Preamble
\documentclass[a4paper,11pt]{article}

% Package Imports
\usepackage[T1]{fontenc}
\usepackage{latexsym}
\usepackage[empty]{fullpage}
\usepackage{titlesec}
\usepackage{marvosym}
\usepackage[usenames,dvipsnames]{color}
\usepackage{verbatim}
\usepackage{enumitem}
\usepackage[pdftex]{hyperref}
\usepackage{fancyhdr}

% Customizations
\pagestyle{fancy}
\fancyhf{}
\fancyfoot{}
\urlstyle{same}
\raggedbottom
\raggedright
\setlength{\tabcolsep}{0in}

% Renewals
\renewcommand{\headrulewidth}{0pt}
\renewcommand{\footrulewidth}{0pt}
\renewcommand{\labelitemii}{$\circ$}

% Adjust margins
\addtolength{\oddsidemargin}{-0.375in}
\addtolength{\evensidemargin}{-0.375in}
\addtolength{\textwidth}{1in}
\addtolength{\topmargin}{-.5in}
\addtolength{\textheight}{1.0in}

% Sections formatting
\titleformat{\section}{
	\vspace{-4pt}
	\scshape\raggedright\large
}{}{0em}{}[\color{black}\titlerule \vspace{-5pt}]

% Custom commands
\newcommand{\resumeItem}[2]{
  \item\small{
    \textbf{#1}{: #2 \vspace{-2pt}}
  }
}
\newcommand{\resumeSubheading}[4]{
  \vspace{-1pt}\item
    \begin{tabular*}{0.97\textwidth}[t]{l@{\extracolsep{\fill}}r}
      \textbf{#1} & #2 \\
      \textit{\small#3} & \textit{\small #4} \\
    \end{tabular*}
    \vspace{-5pt}
}
\newcommand{\resumeSubItem}[2]{
  \resumeItem{#1}{#2}
  \vspace{-4pt}
}
\newcommand{\resumeSubHeadingListStart}{
  \begin{itemize}[leftmargin=*]
}
\newcommand{\resumeSubHeadingListEnd}{
  \end{itemize}
}
\newcommand{\resumeItemListStart}{
  \begin{itemize}
}
\newcommand{\resumeItemListEnd}{
  \end{itemize}
  \vspace{-5pt}
}

%-----------DOCUMENT START---------
\begin{document}

%-----------HEADING----------------
  \begin{tabular*}{\textwidth}{l@{\extracolsep{\fill}}r}
    \textbf{{\Large Ganesh Belgur Ramachandra}} & Email : {ganeshbelgur@gmail.com} \\
    https://www.ganeshbelgur.com/ & Github : github.com/ganeshbelgur \\
  \end{tabular*}

%-----------EXPERIENCE-------------
  \section{Experience}
    \resumeSubHeadingListStart
      \resumeSubheading
        {Sr. Software Development Engineer}
        {Bangalore, India}
        {Advanced Micro Devices (AMD)}
        {Dec 2021 - Present} \\
        \vspace{3pt}
        \textit{\small{https://www.amd.com/en}}

        \resumeItemListStart
          \item\small{{Open source Linux graphics driver development for AMD family of GPUs}}
        \resumeItemListEnd

      \resumeSubheading
        {Software Engineer}
        {Hyderabad, India}
        {Ramp Group, People Tech IT Consultancy Private Limited}
        {Oct 2020 - Dec 2021} \\
        \vspace{3pt}
        \textit{\small{https://www.rampgroup.com/casestudy\textunderscore generalmotors.html}}
        
        \resumeItemListStart
          \item\small{{Implemented an OpenGL ES 3.0/OpenGL 3.3+ based rendering engine from scratch for Blackberry's QNX RTOS and Microsoft Windows, respectively. The engine relies on Qt framework for window management and user inputs from the screens in the automobile.}}
          \item\small{{The rendering engine supports model loading, camera system (with perspective projection), transformations, textures, a primitive material system and a scalable architecture to support other modern graphics APIs such as Vulkan, Metal and DirectX in the future.}}
          \item\small{{Designed complex graphics effects within Altia 3D Design Studio using GLSL shaders for the upcoming instrument cluster gauges of GMC and Chevy vehicles}}
          \item\small{{Successfully implemented a prototype WebSocket server-client mechanism to send instructions from IPC/HMI applications to Unreal Engine to render Ultra Cruise content to screen in real-time. With this prototype, frame rates increased almost 6x when compared to the existing TCP server that relied on long polling.}}
          \item\small{{Implemented the ability to sync the event-loops of Altia 3D and Unreal Engine to enable content shown on screen to be in sync when the two rendering engines are used together}}
          \item\small{{Implemented a quick POC to load compressed textures (KTX format, ETC2 compression) into Mali GPU to play video frames using OpenGL}}
        \resumeItemListEnd

      \resumeSubheading
        {Software Developer - R\&D}
        {Bangalore, India}
        {MPC Film, Technicolor India Private Limited}
        {Feb 2017 - Jan 2020} \\
        \vspace{3pt}
        \textit{\small{https://www.technicolor.com/create/vfx}}

        \resumeItemListStart
            \item\small{{Primarily worked on the refactoring efforts of the in-house procedural fibre-like (hair, fur and grass) geometry generation and grooming tool called Furtility. The tool was used to create visual effects in films and uses a modern curves interpolation algorithm called Catmull-Rom spline interpolation instead of the conventional Beizer curves. The details of these efforts are summarized in a SIGGRAPH paper called "Layering changes in a procedural grooming pipeline" (DigiPro 2018) }}
            \item\small{{Improved preview performance of scenes with high density vegetation by about 40\% with frustum culling }}
            \item\small{{Implemented hero grass wrapping logic based on a research paper titled, "Wires: a geometric deformation technique" (Siggraph 1998) }}
            \item\small{{Refactored the hair generation tool for the new Pixar's \emph{Universal Scene Description} based pipeline at Mill Film }}
            \item\small{{Received multiple on-screen credits in high budget Hollywood feature films for software development: The Darkest Minds (2018), The Lion King (2019), Cats (2019) and Sonic the Hedgehog (2020). (Link: https://www.imdb.com/name/nm10166225/)}}
            \item\small{{GNU Make was the build system and Gitlab was used for CI/CD in a linux environment (CentOS 6) }}
        \resumeItemListEnd

      \resumeSubheading
        {Junior Developer - Pipeline}
        {Bangalore, India}
        {Xentrix Studios}
        {Jul 2016 - Feb 2017} \\
        \vspace{3pt}
        \textit{\small{http://www.xentrixstudios.com/}}

        \resumeItemListStart
          \item\small{{Implemented improvements to digital 3D asset checker and other general quality assurance systems}}
          \item\small{{Wrote a C++ based plugin for Maya that removes millions of duplicated objects in a scene by using a space partitioning technique called Bounding Volume Hierarchy}}
          \item\small{{Supported artists in troubleshooting problematic Maya scenes}}
        \resumeItemListEnd

    \resumeSubHeadingListEnd
%-----------PROGRAMMING SKILLS-----
  \section{Programming  Skills}
    \resumeSubHeadingListStart
      \resumeSubItem{Programming Languages}
        {C++, C, Python, Lua, Javascript;}
      \resumeSubItem{Technologies, Frameworks and APIs}
        {OpenGL, WebGL, Unreal Engine 4, USD, Maya, Katana, Blender;}
      \resumeSubItem{Tools}
        {Git, QMake, GNU Make, Visual Studio, GDB, Renderdoc}
    \resumeSubHeadingListEnd
%-----------EDUCATION--------------
  \section{Education}
    \resumeSubHeadingListStart

      \resumeSubheading
        {Amrita School of Engineering, Amrita Vishwa Vidyapeetham University}{Bangalore, India}
        {Bachelor of Technology in Computer Science and Engineering - 8.13/10.00 (Distinction)}{Aug. 2012 - Aug. 2016}
    \resumeSubHeadingListEnd

%-----------PROJECTS---------------
  \section{Projects - sourcecode available on Github}
    \resumeSubHeadingListStart
      \resumeSubItem{Comet}
        {A unidirectional path tracer implemented based on Peter Shirley's minibook series, PBRT book and various courses by Dr. Károly Zsolnai, Dr. Thomas Auzinger and Dr. Ravi Ramamoorthi. }
      \resumeSubItem{Rosary}
        {A catalogue of modern OpenGL scenes demonstrating perspective projection, free-flying camera system, texture sampling, Phong lighting and shading models, shadow mapping and cube mapping.}
      \resumeSubItem{DreamWorks FX Challenge}
        {A simulation of sparks flying in a projectile trajectory that collides with obstacles and splinters.}
      \resumeSubItem{DreamWorks Steer Quest}
        {A flocking simulation of a herd of sheep avoiding static and dynamic obstacles for a hackathon by DreamWorks Dedicated Unit, India (SIGGRAPH 1987, Craig W. Reynolds)}
    \resumeSubHeadingListEnd

%-----------PUBLICATIONS---------------
  \section{Publication}
    \resumeSubHeadingListStart
      \resumeSubItem{Grammar Error Detection Tool for Medical Transcription Using Stop Words Parts-of-Speech Tags Ngram Based Model}
        {A novel approach in NLP for detecting grammatical errors in a text. A technical paper was presented on this approach at the Jawaharlal Nehru Technological University, Hyderabad and published by Springer.}
    \resumeSubHeadingListEnd

%-----------OTHER ACHIEVEMENTS-----
  \section{Other Details}
    \resumeSubHeadingListStart
      \resumeSubItem{Volunteering}
        {Served as a student volunteer at the ACM SIGGRAPH Asia 2017 (Bangkok), 2018 (Tokyo) and 2019 (Brisbane);
          Volunteer/ Mentor at GAFX 2018, Bangalore;
          Organised various algorithmic hackathons at Amrita; }
      \resumeSubItem{Languages}
        {English (Professional working proficiency); Kannada (Mother tongue);}
    \resumeSubHeadingListEnd

%-----------DOCUMENT END-----------
\end{document}
